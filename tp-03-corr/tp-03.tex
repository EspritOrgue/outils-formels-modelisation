\documentclass[a4paper, titlepage]{article}
\usepackage[round, sort, numbers]{natbib}
\usepackage[utf8]{inputenc}
\usepackage{amsfonts, amsmath, amssymb, amsthm}
\usepackage{color}
\usepackage{framed}
\usepackage{listings}
\usepackage{mathtools}
\usepackage{paralist}
\usepackage{parskip}
\usepackage{subfig}
\usepackage{tikz}
\usepackage{titlesec}
%\usepackage{ulem}

\numberwithin{figure}{section}
\numberwithin{table}{section}

\usetikzlibrary{arrows, automata, backgrounds, petri, positioning}
\tikzstyle{place}=[circle, draw=blue!50, fill=blue!20, thick]
\tikzstyle{transition}=[rectangle, draw=black!50, fill=black!20, thick, minimum width=5.5mm, minimum height=5.5mm]

% define new commands for sets and tuple
\newcommand{\setof}[1]{\ensuremath{\left \{ #1 \right \}}}
\newcommand{\tuple}[1]{\ensuremath{\left \langle #1 \right \rangle }}
\newcommand{\card}[1]{\ensuremath{\left \vert #1 \right \vert }}

\definecolor{lstbg}{rgb}{1,1,0.9}
\lstset{basicstyle=\ttfamily, numberstyle=\tiny, breaklines=true, backgroundcolor=\color{lstbg}, frame=single}
\lstset{language=C}

\makeatletter
\newcommand\deadline[1]{\def\@deadline{#1}}
\newcommand\objective[1]{\def\@objective{#1}}
\newcommand{\makecustomtitle}{%
	\begin{center}
		\huge\@title \\
		[1ex]\small Dimitri Racordon, le \@date
	\end{center}
	\begin{framed}\@deadline\end{framed}
	\begin{framed}\@objective\end{framed}
}
\makeatother

\begin{document}

\title{Outils formels de Modélisation \\ 3\textsuperscript{ème} Travail personnel}
\author{Dimitri Racordon}
\date{27.10.17}

\deadline{
\textbf{Date de rendu}: Jeudi 09.11.17 à 23h55 \\

  Comme leur nom l'indique, ces travaux sont \emph{personnels}.
  La copie est strictement interdite, et toutes similitudes entre deux rendus
  seront santionnées par la note de 0.
  Tout dépassement de la date et heure de rendu sera lourdement pénalisée.
  Date et heure de rendu sont toujours données en heure locale de Genève.
  Tout commit sur votre dépôt publié après la date de rendu sera ignoré.

  Seule la branche \texttt{master} de votre dépôt sera prise en compte
  lors de la correction.

  Votre code doit être compilable en Swift 4.0, avec un compilateur officiel.
  La note the 1 vous sera attribuée si votre code ne peut être compilé.
}

\objective{
  Dans ce TP, vous allez écrire l'algorithme de création d'un graphe de marquage.
}

\makecustomtitle

\section{Mise en place du TP}

Forkez le dépôt https://github.com/cui-unige/outils-formels-modelisation.
Vous travaillerez sur votre propre version du dépôt,
et effectuerez tous vos commit sur ce dépôt-ci.

Le projet pour ce TP se trouve dans le répertoire \texttt{tp-02}.

\section{Construction du graphe de couverture}

La méthode \texttt{PTNet.coverabilityGraph(from:)}
(dans le fichier \texttt{PTNet+Extensions.swift})
est supposée retourner le graphe de couverture d'un réseau Pétri,
d'après un marquage initial donné.

Ecrivez l'implémentation de cette fonction.

\end{document}
